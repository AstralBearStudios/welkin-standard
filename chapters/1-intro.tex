% SPDX-FileCopyrightText: 2023 Oscar Bender-Stone <oscarbenderstone@gmail.com>
% SPDX-License-Identifier: CC-BY-4.0
% 1-intro.tex - Introduction for Welkin Book

% TODO Add proper bibtex references

\chapter{Introduction}
\label{ch:intro}

% Starting point for Deleuze: https://plato.stanford.edu/entries/deleuze/ https://link.springer.com/chapter/10.1057/9780230280731_11

Startng Outline:
\begin{itemize}
  \item Describe the ways philosophers and mathematicians have approached organizing the world around them.
  \item Reflect on two major shortcomings of these approaches: having little metaphysical explanation about \textit{how} things can be combined, and a lack of generality for several concepts
        \begin{itemize}
      \item No explanation for what \textit{is} a set fundamentally, or, more importantly, how two entities can be combined into one.
      \item Lack of generality for: structures, the real numbers, logic
  \end{itemize}
  \item Explain the essential goals of Welkin, built upon CFLT
  \item Describe target audience
  \begin{itemize}
    \item This document is geared towards philosophially or mathematically inclined persons.
    \item In a different document (separate from this repo), there will be a more accessible version as a part of my program on ``humanistic logic''.
  \end{itemize}
\end{itemize}
