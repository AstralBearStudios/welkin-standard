% SPDX-FileCopyrightText: 2023 Oscar Bender-Stone <oscarbenderstone@gmail.com>
% SPDX-License-Identifier: MIT
% 4-spec.tex - Specifices the Welkin language and provides a theoretical framework
% for general implementations

% TODO: create a version for welkin-standard.tex that directly includes the copyright notice
% Henceforth, compliant Welkin implementations will be collectively referred to as ``Welkin.'' A formal definition of

This document describes the Welkin information language, a programming language aimed and preserving, analyzing, and extending information.


This edition of the standard, in English, is the basis for all other translations. Only the grammars will be given in an English (specifically ASCII) and should be copied identically. However, these grammars can be built upon via Section ?.?, and any other terms in this document may be translated or changed as necessary.

Throughout this document, every instance of ``Welkin grammar'' means ``standard Welkin grammar.'' Every instance of ``Welkin interpreter'' means ``conformant Welkin interpreter.'' For a definition of conformance, refer to \ref{section: conformance}.

\section{Preliminaries}

% Helpful macros for terms for character encoding in math mode
\newcommand*{\chars}{\mathrm{CHAR}}
\newcommand*{\numbers}{\mathrm{NUMBERS}}
\newcommand*{\whitespaces}{\mathrm{WHITE\_SPACES}}
\newcommand*{\reserved}{\mathrm{RESERVED}}
\newcommand*{\strings}{\mathrm{STRING}}
\newcommand*{\term}{\mathrm{term}}
\newcommand*{\terms}{\mathrm{terms}}
\newcommand*{\delimiters}{\mathrm{DELIMITERS}}
\newcommand*{\escapes}{\mathrm{STRING\_ESCAPES}}
\newcommand*{\encoding}{\mathcal{E}}
\newcommand*{\decoding}{\mathcal{D}}
\newcommand*{\can}{\textrm{Can}}


\newcommand*{\scope}{\textrm{scope}}

\SetBNFConfig{
  relation = {::=|=>},
  relation-sym-map = {
    {::=} = {=},
    {=>} = {->},
    {<~} = {},
  }
}
\subsection{Mathematical Background}
This standard requires a cursory background in discrete mathematics, parsing, and Backus-Naur Form (BNF). A reading of [] and [] suffices to understand this document. We clarify our mathematical notation below.

Let $A, B, C$ be sets and $n \in \mathbb{N}$. We define $[n] := \{0 \leq k \leq n-1\}.$ We denote the disjoint union of $A, B$ as $A \sqcup B,$ and an arbitrary equivalence relation on $A$ by $\sim_{A}.$ We let
$A^{*}$ be the set of finite sequences of $A$ (or \texttt{A*} in text files), including the empty sequence $\varepsilon := \emptyset$ (\texttt{empty} in text), and we call each $w \in A^{*}$ a \textbf{word.}

Let $f: A \to B,$ $g: B \to C$ be functions. We denote the composite $g \circ f$ as the function $g \circ f = g(f(x)),$ and by $f^{*}: A^{*} \to B^{*}$ the extension of $f$ to finite sequences, given by $f^{*}(a_{1}, \cdots, a_{n}) = (f(a_{1}), \cdots, f(a_{n})).$ We frequently use the product and disjoint union of two functions. We explicilty define these for a family of sets $A = \{A_j\}_{j \in J}$, with $|J| = n.$\footnote{These are all examples of universal properties found in category theory; see \cite{maclane} for further information. It suffices to expand these definitions in certain cases for this standard.}
\begin{itemize}
\item For the carteisan product $P = \prod_{j \in J}A_{j},$ there are \textbf{projections} $\pi_j: P \to A_{j},$ where $\pi_i(x_{1}, x_{2}, \cdots, x_{n}) = x_{i}.$ For $n$ functions $f_{j}: A_{j} \to C,$ there is a unique function $\prod_{j \in j}f_{j} := f: P \to C$ such that $f = f_{j} \circ \pi_{j}.$
  \item For the disjoint union $D = \prod_{j \in J}A_{j},$ there are \textbf{injections} $\i_{j}: A_{j} \to D,$ where $i_{j}(x) = (x, j).$  For $n$ functions $g_j: C \to A_{j}$ there is a unique function $\coprod_{j \in J}g_{j} := g: C \to D$ such that
$g = i_{j} \circ g_{j}.$
\end{itemize}

Finally, except when explicitly stated, all statements with a free variable $x$ are universal, i.e., must hold for all $x \in X.$
% % TODO: use this definition with sets index by I \times J
% \begin{itemize}
%   \item

%   % \item For any disjoint union $A \uplus B,$ there are disjoint unions $i_{A}: A \to A \uplus B, i_{B}: B \to A \uplus B$ given by $i_{A}(a) = a, i_{B}(b) = b.$ To define $f: A \uplus B \to C$ means providing functions $f_{1}: A \to C, f_{2}: B \to C$ such that
%   % \begin{align*}
%   %   f \circ i_{A} = f_{1}, f \circ i_{B} = f_{2}.
%   % \end{align*}
%   % \item For any cartesian product $A \times B,$ there are projection functions $\pi_{1}: A \times B \to A, \pi_{2}: A \times B \to A$ given by $\pi_{1}(a, b) = a, \pi_{2}(a, b) = b.$ To defines $g: A \to B \times C$ means providing functions $g_{1}: B \to A, g_{2}: C \to A$ such that
%   % \begin{align*}
%   %   \pi_{A} \circ g = g_{1}, \pi_{B} \circ g = g_{2}.
%   % \end{align*}
% \end{itemize}

% TODO: generalize McAllester's definition to include set theory (or reference type theory, as necessary)
Finally, two structures are \textbf{cryptomorphic} if their key properties are defined by the same first order theory. % Any structure in the standard may be considered modulo cryptomorphism. Thus, any implementation can use their own data structure, \textit{as long} as it satisfies the properties in this document.

\subsection{Character Encodings}
% TODO: generalize to byte encodings and numbers.
We define text, character encodings, and character decodings as abstract notions. The discussion here may be carried out in terms of bytes and specific data formats, but these concepts are beyond the scope of this standard.

Let Char be a finite set. An \textbf{encoding} is an injective mapping $\encoding : \mathrm{Char} \to \mathbb{N}.$ The associated \textbf{decoding} is the left-inverse $\decoding: \encoding^{-1}(\mathbb{N}) \to \mathrm{Char}$ of $\encoding.$ We denote $\chars = \decoding(\mathbb{N}).$

Character encodings may be given as finite tables, matching natural numbers with characters. Several major encodings are defined in the following sources.
% TODO: put these references in bibtex
% US-ASCII (see references therein): https://www.iana.org/assignments/character-sets/character-sets.xhtml
% UTF-8, UTF-16: https://www.unicode.org/versions/Unicode15.0.0/UnicodeStandafinite rd-15.0.pdf
\begin{itemize}
	\item US-ASCII []. We will refer to this simply as ASCII\footnote{American Standard Code for Information Exchange}, but there are subtle variations accross specific nationalities and applications (see []).
	\item UTF-8, UTF-16 []. The Unicode Standard defines encodings across numerous human languages and unique characters.  % TODO write notes about both
\end{itemize}

% We render special characters, frequently used in our BNF, as glyphs. See Table ?.? for more details.

Although ASCII is a subset of UTF-8, this standard will prioritize ASCII as much as possible. The BNFs for this standard (\ref{section:grammar}) are written in ASCII as a unifying encoding, but users may create grammars using UTF-8, UTF-16, or their own encodings (see \ref{section:customization}).
% TODO: put this into table format. We need an easy way to reference any of these sets and notes about them
% Defining each subimage linearly is not working that well.


% TODO: decide if BNF encoding is a good name. Maybe something to do with terminals?
% TODO: decide if string escapes is parametric or must be consistent
% TODO: allow numbers to be delimtiers (in case users want that)
\begin{definition}
  A \textbf{BNF encoding} under Char is an encoding $\encoding: \textrm{Char} \to \mathbb{N}$ along with five subsets $\chars*,$
  \begin{itemize}
  \item $\whitespaces$
  \item $\numbers$
  \item $\delimiters$
  \item $\escapes$
  \item $\reserved$
  \end{itemize}
  such that $\whitespaces$ is pair-wise disjoint with every other subset.

Moreover, let $\strings$ be the set of strings over $\delimiters,$ where a string is a word $d_{1}wd_{2}$ such that $w \in (\chars \setminus \delimiters \cup \escapes(d_{1}, d_{2}))*.$ Notice that $\delimiters, \escapes$ can be defined in terms of $\strings.$
\end{definition}
In our BNF encoding, we assume all sets above are non-empty. The first is an optional addition, adding support for defining machine representable numbers; the second distingushes words from one another; the next two are used for arbitrary strings; and the last enables parsers to priortize certain characters. Any of these sets, however, can made empty with a different grammar (see Section ?.?).

% \begin{definition} A \textbf{BNF encoding under} Char is an encoding $\mathcal{E}: \mathcal{N} \to \textrm{Char}$ along with three disjoint subsets.



% \end{definition}

% Let $\escapes \subsetq \Char{*} \setminus W.$ This is optional and enables a grammar to include an existing literal in
% TODO: decide if this should be a separate definition
Every Welkin grammar, written in a BNF metasyntax, uses the ASCII encoding with
\begin{itemize}
  \item $\numbers = \emptyset$,
  \item $\whitespaces = $ = ASCII whitespace characters,
  \item STRING = all ASCII characters enclosed by single or double quotes. We escape single (double) quotes via \' (\textbackslash'').
\end{itemize}

% We list several distinguished subsets secondary notions in Table ?.? All of these sets, except $\strings(d)$ and $D_{\strings},$ are arbitrary.
% % TODO: Remove member notation? It seems like we don really need it
% \begin{center}
%   \bgroup
%   \def\arraystretch{2.0}
% \begin{tabular}{| c | c | c |}
%   \hline
%   \textbf{Set} & \textbf{Definition} & \textbf{Member Notation} \\
%   \hline
%   NUMBERS & Subset of $\chars*$ & $r$ \\
%   \hline
%   WHITE\_SPACES & Subset of $\chars*$ & $ws$ \\
%   \hline
%   DELIMITERS & Subset of $(\chars*)^{2}$ & $d = (d_{1}, d_{2})$ \\
%   \hline
%   STRING($d$) & \makecell{$s = d_{1}wd_{2},$ $d_{1}, d_{2} \not \subseteq_{ws} w.$ \\ $w$ is the \textbf{contents of} $s$ } & \makecell{$s_{d},$ with \\ contents $\hat{s}_{d}$} \\
%   \hline
%   STRING & Subset of $\bigcup \strings(d)$ & \makecell{$s,$ with \\ contents $\hat{s}$} \\
%   \hline
%   $D_{\strings}$ & \makecell{Set of delimiters \\ appearing in $\strings$} & \makecell{$d_{\strings}$} \\
%   \hline
%   ESCAPE$(W)$ & Subset of $\chars*\setminus W$ & $escape_{W}$ \\
%   \hline
%   ESCAPE & Subset of $\bigcup \escapes(d)$ & $escape$ \\
%   \hline
% \end{tabular}
% \egroup
% \end{center}

% Strings may include their delimiters by using $\escapes(\{d_{1}, d_{2}\})$ as a prefix or suffix.

% We implicitly assume that does not conflict with literals defined in a grammar. In terms of the recommended LALR parser, this means that literals are matched first, not identifiers. However, these characters may be used by creating a custom grammar (see Section ?).

% A { { A {} }

A \textbf{text} is a subset of $\chars*.$ We will not consider streaming issues, i.e, we will assume every Welkin file is present at one time.
% TODO: describe issues with files! How to work with this?

\subsection{BNF Variant}
Our variant of BNF uses the notation shown in Table ?.? and in Definition ?.?. Our notation, as well as every standard Welkin grammar, is available in ASCII.

Each BNF has an associated subset $\reserved \subseteq \chars*$ for any literals that appear in the grammar. These literals are always matched first, which means that set-based terminals (e.g., from $\numbers$) . We will explictly state these for the Welkin variants in the next section.

% TODO: fix spacing issues with all rows. BNF is not fitting in there well!
\begin{center}
  \begin{tabular}{ | c | p{2cm} | p{6cm} | }
  \hline
  \textbf{Concept} & \textbf{Notation} & \textbf{Example} \\
  \hline
  Rule Assignment & $=$ & \begin{bnf} term ::= atom\end{bnf}\\
  \hline
  \makecell{Empty Word \\ (In ASCII)} & empty & \begin{bnf} term ::= empty\end{bnf}\\
  \hline
  Alternation & $|$ & \begin{bnf} boolean ::= true | false \end{bnf}\\
  \hline
  \makecell{Concatenation \\ (No white spaces \\ inbetween rules)} & \makecell{Separate with \\  a period (.)}&  \begin{bnf} function ::= name.``(''.number.``)''\end{bnf} \\
  \hline
  \makecell{Concatenation \\ (Zero or more \\ white spaces allowed)} & \makecell{Separate with \\ white space} & \begin{bnf} data ::= date name\end{bnf} \\
  \hline
  Groupings & Parantheses () & \begin{bnf} boolean ::= (true // false)\end{bnf} \\
  \hline
  Literals & ``word'' & \begin{bnf} boolean ::= ``true'' // ``false'' \end{bnf} \\
  \hline
  Choice Names & terms $\to$ rule & \makecell{\begin{bnf} boolean ::= ``true'' $\to$ true || ``false'' $\to$ false \end{bnf}, \\ equivalent to \\ \begin{bnf}  boolean ::= true // false ;; true ::= ``true'' ;; false ::= ``false'' \end{bnf}} \\
  \hline
  \makecell{Rule Substitution \\ (Definition ?.?)} & rule1[rule2 $\to$ rule3] & \makecell{\begin{bnf} rule1 ::= rule2 \end{bnf}, \\ equivalent to \\ s } \\
    \hline
  \end{tabular}
  \end{center}

We introduce one new notion that is definable with BNFs.
\begin{definition} (Rule Substitution) Let $rule_{1}, rule_{2}$ be rules with $rule_{2}$ appearing on the right hand side of $rule_{1}.$ Suppose $rule_{3}$ appears on the lefthand side of $rule_{2}.$ Then $rule_{1}[rule_{2}\to rule_{3}]$ is $rule_{1}$ with every instance of $rule_{2}$ replaced by $rule_{1}.$

% TODO: clean up this definition!
Some particular cases:
\begin{itemize}
\item $term[empty]$ means no rule should be applied.
\end{itemize}

\end{definition}

An conformant parser for all three grammars should include the ability to compose and override rules. This ensures that any updates to grammars, if necessary, are isolated. See Section ?.? for more details.
% TODO: convert this into a table for easy access
% \begin{itemize}
%   \item \texttt{term ::= $S \subseteq \chars^{*}$} means \texttt{term $\in S$},
% or most one instances and is shorthand for
% 	\begin{center}
% 		\texttt{terms ::= term terms | term},
% 	\end{center}
%
\section{The Welkin Language}

There are three fundamental variants of Welkin that define the foundation for the language:
\begin{itemize}
	\item Base Welkin, mirroring the key properties of the core data structure.
	\item Attribute Welkin, extending Base Welkin with attributes. Attributes are a limited type of directive that can customize how the interpreter accepts or presents data.
	\item Binder Welkin, enabling arbitrary evaluation of Welkin files and access to the user's operating system. This is equivalent to Attribute Welkin with three new directives: \texttt{@eval}, \texttt{@exec}, and \texttt{@bind}. \end{itemize}
Each of these variants can be parsed with LALR parsers and fundamentally have the same semantics. However, in Binder Welkin, \texttt{@eval} makes the interpreter Turing complete (see Section ?.?), and using \texttt{@exec} can run external programs that impact the user's system. For this reason, Binder Welkin is a separate, optional component, as detailed in Section ?.?.
%\renewcommand{\syntleft}{\normalfont\bfseries}
%\renewcommand{\syntright}{}

% TODO: There should be some consistent procedure to ensure that the BNF here
% is the same as those found in any implementation
%\renewcommand{\bnfexpr}{\textbf}
\subsubsection*{Syntax}
\label{section:syntax}
Define each $\reserved$ set as follows.\footnote{In with single characters, we write them with concatenation (possibly spaces inbetween). For example, $\{\texttt{x y}\}$ stands for $\{\texttt{x, y}\}.$}
% TODO: figure out better way to display these! Hard to read delimiters!
\begin{itemize}
\item $\reserved_{\textrm{base}} = \{\texttt{\{\} () [] . =}\} \cup \{\texttt{->, <-, \_\{},\} $
  \item $\reserved_{\textrm{attribute}} = \{\texttt{\{\} () [] . , * @}\} \cup \{i\texttt{--, ->, <-}\}$ % TODO: remove commas for single characters for clarity
\item $\reserved_{\textrm{binder}} = \reserved_{\textrm{attribute}}.$
\end{itemize}
Each grammar is provided in Table 3.?.
% TODO; figure out rule for when NUMBERS is empty
% TODO: make different RESERVED key words for Base and Attribute Welkin. Binder Welkin is fairly straightforward
% (We actually only need to look at prefixes; any attribute names WILL be parsed first. Welkin will assume
% the user wanted the built-in directives. A different name for those should be used)
% TODO: It may be better to type up Welkin in regular font; it looks readable and could be done well in math mode.

% \SetBNFConfig{

%   relation-sym-map = {
%     {::=} = {\ensuremath{=}},
%   }
%   }
%     TODO: Remove Notes column
% TODO: add -> back into the grammar; using -> directly conflicts with simplebnf
\begin{table}[hbt!]
  \centering
    \begin{tabular}{| p{1.5cm} | p{9.5cm} |}%{1\textwidth}{| l | X |}
    \hline
    Variant & Grammar \\\hline % & Notes \\
      \makecell{Base \\ Welkin} &
  \begin{bnf}
  terms ::= term* ;;
  term ::= (graph // connections // member // unit) ;;
  graph ::= (unit ``:='')? ``\{'' terms ``\}'' ;;
  connections ::= term (connector term)+ ;;
	connector ::=
   | ``-'' term ``-'' $\to$ edge
   | ``-'' term ``>'' $\to$ left\_arrow
	 | ``<-'' term `-' $\to$ right\_arrow ;;
  member ::= unit? (``.''.(ident // string)? // ``\#''.num )+ ;;
	unit ::= ident // string // num ;;
  ident ::= CHAR* ;;
	string ::= STRING ;;
	num ::= NUMBER
\end{bnf} \\ %& If $\numbers = \emptyset,$ any instance of \texttt{num} should be removed from the parser. \\
   \hline
      Attribute Welkin &
  % TODO: decide how to make import and override statements a part of bnf
  \makecell{\%import grammars/base.txt \\ \%override term}
  \begin{bnf}
  term ::= ``@''.(directive // graph[directive]) | construct // graph // connection | member // unit ;;
  directive ::= attributes ;;
  attributes ::= ``import''.tuple $\to$ import
  | ``self''.(member?) $\to$ self
  | ``alias''.graph[empty] $\to$ alias
  | ``extend''.graph[empty] $\to$ extend
  | ``resource''.graph[unit] $\to$ resources
  | ``metadata''.graph[unit] $\to$ metadata
  | ``input''.graph $\to$ input
  | ``parse''.(graph // unit) $\to$ parse
  | ``validate''.tuple $\to$ validate
  | ``record''.term $\to$ record
  | ``print''.graph $\to$ print
  | ``attribute''.graph $\to$ new\_attribute
  | unit.term $\to$ custom\_attribute ;;
  construct ::= operation // tuple // list // series // all\_terms ;;
  operation ::= term.tuple // term unit term ;;
  tuple ::= ``('' series ``)'' ;;
  list ::= ``['' series ``]'' ;;
  series ::= term ``,'' (term ``,'')* term ``,''? ;;
  all\_terms ::= ``*''
 \end{bnf} \\
                      %& This is limited to the CLI and GUI (TBD). \\
   \hline
   Binder Welkin &
\makecell{\%import grammars/attribute.txt \\ \%override directives}
    \begin{bnf}
     directives ::= attributes // binders ;;
     binders ::= ``eval''.tuple[unit] $\to$ eval
     | ``exec''.tuple[string] $\to$ exec
     | ``bind''.graph[empty] $\to$ bind
   \end{bnf} \\
    \hline
    \end{tabular}
  \end{table}


% Throughout this document, Welkin documents are formatted with the following convention: the ASCII sequence \texttt{->} is rendered as $\to$ (A graphical user interface may support this rendering via glyphs). % TODO: put special renderings in a table
  % TODO: figure out suitable grammar composition notation

% TODO: recognize, in cflt, that the above welkin file is in fact a context free grammar! The more important part, which we need to still define, is the semantics, which will have its full strength with the full grammar. (Maybe we should change that option to be semantics instead?)

\subsubsection*{Semantics}
We break down our semantics first by terms. Directives are handled separately in the next section.
% TODO: make this more efficient. (Probably best to follow the structure from a logic textbook)
\begin{definition}
Equality of terms.
\begin{itemize}
    % TODO: decide if ''A`` is the same as A
  \item \textbf{Basis.} Two units are equal if they are the same kind and obey one of the following.
	\begin{itemize}
	  \item \texttt{ident} terms are equal if their corresponding characters are equal,
    \item \texttt{string} terms are equal if their corresponding contents are equal.
	  \item \texttt{num} terms are equal if they represent the same value. Thus, \texttt{1} coincides with \texttt{10E}.
	\end{itemize}
  \item \textbf{Recursion.}
        \begin{itemize}
        \item Two members are the same if they contain the same list of units. % TODO: include the case of relative imports
        \item Two connectors are equal if they are equal as terms. %TODO: rework to make this clearer
 		  \item Two connections are equal if they connect the same terms and have equal connectors.
		  \item Two graphs are equal if they contain the same terms.
	\end{itemize}
\end{itemize}
\end{definition}
% This definition should be unnecessary; this should be clear from the BNF.
% \begin{definition} (Membership) Let $t$ be a term and $G$ a graph. We say $t$ is a \textbf{member} of $G$ if $t$ appears as the contents of $G.$\end{definition}
% \end{definition}
A \textbf{scope} is recursively defined and intutively is a level of terms.
\begin{definition} (Scope)
 Let $t$ be a term.
  \begin{itemize}
	\item If $t$ is not contained in a graph, then $\scope(\texttt{term}) = 0,$
	\item If $G' \in G$ are both graphs and $\scope(G') = n,$ then for all $t \in G',$ $\scope(t) = \scope(G) + 1.$
\end{itemize}
\end{definition}
A \textbf{valid} base Welkin file consists satisfies a unique naming rule: in every scope, there are no name collisons. In particular, every graph must \textbf{only be defined once.} Note that, by the way equality was defined between two numbers,
  there can only be one representation of a given number in a scope. For example, using \texttt{1} and \texttt{10E-1} in the same scope would produce a name collison.
\\ We first form an Abstract Syntax Tree (AST), from which we form the final stored data in a \textbf{Welkin Information Graph.}
\begin{definition}
  Base Welkin is parsed into the following AST $\mathcal{A}.$
  \begin{itemize}
	\item Every term is a new subtree with its contents as children.
    \item Every graph is an ordered pair of its name and list of children.
    % TODO: how to deal with empty connectors? Big issue for parsing and storing in a WIG!
    \item Every connection is an ordered pair:
		  \begin{itemize}
			\item Left arrows $u \xrightarrow{e} v$ correspond to a triple $(u, e, v);$
			\item Right arrows $u \xrightarrow{e} v$ correspond to the triple $(v, e, u);$
			\item Edges correspond to both a left and right arrow.
		  \end{itemize}
	\item Every unit is encoded via $\encoding^{*}.$ Numbers are further transformed into a machine representable form, which is dependent on the implementation.
  \end{itemize}
 \end{definition}
 % TODO: talk about encoding of numbers in the structure itself. This probably a separate encoding from the one used to write the welkin file

 We adapt the terminology from Jensen and Milner \citep{jensen-milner-bigraphs} for our needs.

% \begin{definition} A \textbf{labeling signature} $\mahcal{L} = (L, p_{L})$ consists of
%    \begin{itemize}
%      \item a set $L$ of \textbf{labels,}
%     \item a function $L$
%    \end{itemize}
% \end{definition}

% TODO: revisit bigraph isomorphism and determine if we could use an adjaceny map instead. (Parent function forces us to define roots; this is not needed with child based approach
% INSIGHT: I thought we needed to add a tree based structure to the labels (to make sure that user's content is preserved), BUT we can add this later. In fact, we can put this
% into our canonical form! (Can(G) needs to ensure we CAN recover related labels; we'll define this formally, but essentially, if parent(1) = 0, 1 is labeled with a, 0 is labeled with A, then
 % we have to recover this original labeling.) We stil need to figure out how to merge multiple labelings together...
\begin{definition}
  % A unit graph is a tree. A labeled unit graph is a tree whose nodes are labeled by trees.
  A \textbf{unit graph} $U = (V, V^{\top}, p)$ consists of
  \begin{itemize}
    \item a set $V$ of \textbf{units,}
    \item a function $p: V_{0} \subseteq V \to V$ called the \textbf{parent function,}
  \end{itemize}
  such that $p$ is acyclic: $p^{(k)}(v) = v$ iff $k = 0$. We define

  \begin{itemize}
  \item $V^{\top} = V \setminus V_{0}$ as the set of \textbf{roots,} and
  \item $V^{\bot} = \{v \;:\; p^{(k)}(v) \neq u \}$ as the set of \textbf{sites.}
  \end{itemize}
  Equivalently, $U$ is a forest, where each component has some root in $V^{T}$ and leaves in $V^{\bot}.$
  % TODO: figure out if we need to say that L is isomorphic to U. This may ensure that L doesn have too much of a different structure than U
  A unit graph $\mathcal{L} = (N = L / \sim_{L}, p_{\mathcal{L}})$ is said to \textbf{label} a unit graph $U = (V, p)$
  if there exists a function $l: V \to N \sqcup \varepsilon$ such that, whenever $l(u) = l(v),$ $l(v) = \varepsilon$ or $p(u) = p(v), l(v) \neq \varepsilon$ imply $u = v.$ In this case, we call $\sim_{L}$ an \textbf{alias equivalence} and $l$ a \textbf{labeling.} In general, $L = \coprod_{1 \leq j \leq m}L_{j}$ for finitely many labels $j.$

  % In other words, the restriction of $l$ to $\{(u, v) \;:\; p(u) = p(v)\} \cup l^{-1}(\varepsilon)$ is injective.
   % \footnote{In order to support anonymous graphs, it is necessary to add the empty set as a label. The special labeling condition ensures that vertices are uniquely determined by non-empty labels.}
\end{definition}

\begin{definition}
  A \textbf{connection graph} $G = (V, C)$ consists of
  \begin{itemize}
    \item a set $V$ of \textbf{vertices,} and
    \item a set $C \subseteq V^{2} \cup V^{3}$ of \textbf{connections.} % TODO: re-evaluate if V^2 \cup V^3 OR V \times (V \cup \varepsilon) \cup V should be used!
  \end{itemize}
  Connections in $V^{2}$ are called \textbf{atomic connections} or \textbf{arcs.} A connection graph is \textbf{reflexive} if $(A, A, A) \in V$ for each $A \in V.$

  We frequently use the \textbf{component} functions of $C,$ given as follows.
  \begin{itemize}
  \item The \textbf{source map} $s = \pi_{1} \sqcup \pi_{1}: C \to V$ returns the first entry in any connection,
  \item The \textbf{target map} $t = \pi_{2} \sqcup \pi_{2}: C \to V,$ returns the second entry in an arc, or the third element in a non-atomic connection, and
  \item The \textbf{connector map} $c = \pi_{2}: C \cap V^{3} \to V$ returns the second entry in a non-atomic connection.
  \end{itemize}
\end{definition}

% We frequently use an auxilary notion known as a port, as used by \cite{milner}.

% \begin{definition}
% A \textbf

% \end{definition}

%Throughout this document, we will assume all connection graphs are simple.
%\footnote{It is possible to allow loops, i.e., connections of the form $(A, A)$ or $(A, A, A).$ This property, however, is automatically part of a graph's structure: a morphism \ref{def:morphism}
%maps vertices of one graph to another, while separately mapping connections to connections. Separating these maps ensures that a non-trivial graph cannot be embeded into a single point, avoiding structural collapse. }

% In this case, there is a function $i: V \to C$ given by $i(A) = (A, A, A).$ Abusing notation, we will write $A$ and $(A, A, A)$ synonymously (with the notation being clear from context).

Now we may present the core data structure of Welkin (modulo cryptomorphism).

\begin{definition} A \textbf{Welkin Information Graph (WIG)} $G = (V, p, C, L)$ consists of
  \begin{itemize}
    \item a unit graph $(V, p),$ with labels $L,$
    \item a reflexive connection graph $(V, C).$
  \end{itemize}
\end{definition}

WIGs are a special case of a (lean) bigraph \cite{jensen-milner-bigraphs}, in which there is a ``site'' for each element in $V^{\bot},$ a ``root'' for each $v \in V^{\top},$ and there are no inner or outer faces. We add an independent labeling component that does not impact the overarching structure, but directly corresponds to a user's naming conventions.


An important aspect of bigraph theory is Jensen and Milner's complete algebraic characterization \cite{jensen-milner-bigraphs}. This standard only requires one aspect of this theory.

\begin{lemma}
WIGs are closed under disjoint unions.
\end{lemma}
\label{lemma:closure}

Lemma \ref{lemma:closure} ensures that WIGs may have an arbitrary size without special constraints. In fact, any new additions to the graph does not remove existing structure (see \ref{lemma:struct}). It is possible for existing structure to be repeated, in which case, the complexity of the connection graph does not increase (see \ref{tests:examples} for examples).

% \begin{proof}
% Trees and graphs are closed under each operation above, so it suffices to prove the lemma for the labels of a WIG. This follows by straightforward constructions.
% \end{proof}

% TODO: talk about whether edges can collapse into loops. This is not possible in an isomorphism, but it is worth mentioning

% \delta_


To support the identification of equivalent structures, we introduce a WIG equivalence, a special case of lean support equivalence in \cite{jensen-milner-bigraph}.
\begin{definition}
  Let $G = (V_{G}, p_{G}, C_{G}, \{L_{G, i}, l_{G, i}\}), H = (V_{H}, p_{H}, C_{H}, \{L_{H, i}, l_{H, i}\})$ be WIGs. A \textbf{morphism} $f: G \to H$ consists of functions $(f_{V}: V_{G} \to V_{H}, f_{C}: C_{G} \to C_{H}, f_{L}: L_{G} \to L_{H})$ such that the folowing equalities hold:
  \begin{itemize}
    \item $p_{H} \circ f_{V} = f_{V} \circ p_{G},$
    \item Connections are preserved:
    \begin{itemize}
      \item $(A, B) \in C$ implies $(f(A), f(B)) \in C$
      \item $(A, e, B) \in C,$ $(A, e, B) \in C_{G}$ implies $(f_{V}(A), f_{V}(e), f_{v}(B)) \in C_{H}.$
    \end{itemize}
    Equivalently, the following equalities hold.
    \begin{itemize}
      \item $s_{H} \circ f_{C} = f_{C} \circ s_{G},$
      \item $t_{H} \circ f_{C} = f_{C} \circ t_{G},$
      \item $c_{H} \circ f_{C} = f_{C} \circ c_{G}.$
    \end{itemize}
    % \item $f_{V}$ preserves edges: $f_{C}(A, e, B) = (f_{V}(A), f_{V}(e), f_{V}(B)) \in C$ for $A, e, B \in V_{G},$. Equivalently, $(A, e, B) \in C_{G}$ implies $(f_{V}(A), f_{V}(e), f_{V}(B)) \in C_{H}.$\footnote{Note that it is possible for loops to be sent to edges. In an isomorphism, however, this is not possible: all loops are mapped exactly to loops.}
    \item Preservation of loops: for $D \in V_{H},$ $f_{C}^{-1}(D, D, D) \subseteq \{(A, A, A) \in V^{3}_{G}\},$
    \item $l_{H, i} \circ f_{V} = f_{L} \circ l_{G,i}.$
  \end{itemize}
  An \textbf{equivalence} is bijective morphism. We write $G \cong H$ whenever two WIGs are equivalent.
\end{definition}
Our definition excludes the notion of ports, which are not needed in this standard. More information can be found in \cite{jensen-milner-bigraph}.



% TODO: reword this. Graph isomorphism is getting repetitive here.
% TODO: cite specific definition in paragraph below (for support equivalence)
Note that support equivalence includes a bijective graph homomorphism which does not induce a bijection between edges, but it does induce an equivalence relation \cite[jensen-milner-bigraph].\footnote{In category theory, equivalences are formally defined using maps between maps, and typically arise through adjunctions of categories (see [?]).} Thus, while graph isomorphisms are support equivalences (using trivial tree like-structures), the converse is not true.\footnote{See \cite{iso-counterexample}, Figure ?.? for a counterexample. Each graph is equivalent to a bigraph where all the nodes are roots.} This fact is crucial for Welkin implementations: there is a polynomial time algorithm for bigraph equivalence\cite{canonical-string}, while no such algorithm is known for graph isomorphism \cite{babai}. Determining WIG isomorphism is a core component of Welkin (see \ref{sec:core}), and we address how users may work with general graph equivalences (refined or coarse) in \ref{sec:customization}.

% In \cite{milner}, the equivalence classes of bigraphs modulo support equivalence are called abstract bigraphs.

In order to store WIGs for efficient retrieval, we define the associated canonical form, which is parameteric over the representation of nodes.

% TODO: maybe we could generalize the set of vertices chosen? They are pretty arbitrary, but can always be represented by vertices
% TODO: reword the canonical mapping on vertices. This is currently a bit messy.
% TODO: look up storing topoi in a computer; they do use [n] for the representation of graphs!
\begin{definition}
  Let $G$ be a WIG. The \textbf{Welkin Canonical Form (WCF) of} $G$ is the WIG $\can(G) = (\can(V), \can(p), \can(C), \{L_{i}, \can(l_{i})\}),$
  where
  \begin{itemize}
    \item $\can(V) = [|V|],$
    \item Each $v \in V$ is assigned to a unique $\can(v) \in \can(V),$
    \item $\can(p)(\can(v)) = \can(p(v)),$
    \item if $(A, B) \in C,$ $\can(A, B) \in \can(C),$ and likewise, $\can(A, e, B) $
    \item $\can(l_{i})(\can(v)) = l_{i}(v).$
  \end{itemize}
In case $f(n) = [n],$ we say that the canonical form is ``natural.''
\end{definition}
\label{def:iso}

We summarize some key properties in the next lemma, which immediately follows by \label{def:iso} and \label{lemma:closure}

\begin{lemma}
  For any WIGs G, H, and fixed representation of nodes for $\can(G),$
  \begin{itemize}
    \item $G \cong \can(G),$
    \item $G \cong H$ iff $\can(G) = \can(H),$ and
    \item for any labels $L'_{i}, L''_{i},$ $G \sqcup L'_{i} \cong H \cup L^{''}_{i}.$
  \end{itemize}
\end{lemma}

The next lemma defines the basis for the recorder. Every WIG must be put into a universal form that can be labeled in completely different ways. The underlying structure of the WIG is always preserved, no matter what labels are added.
% TODO: we do NOT want to make labels required in this definition. That's the whole point! So, our isomorphism should NOT be reliant to ANY set of labels. That may need to be a looser definition. It would be much easier to consider graphs MODULO these labels. And THEN we can apply WIG isos
\begin{lemma} For any WIG $G,$ $G \cong \can(G).$
\end{lemma}

% % TODO: in the conversion, note that A - A - B -> B, equivalent to A --> B, is converted to A - { A - B } -> B. So both of them are equivalent on Welkin. Every other connection, however, will be parsed with full brackets, i.e.,
% % A - {{ A --> B }} -> B will be parsed as it is.
% \begin{lemma}
%   The conversion from ASTs to Welkin Information Graphs is valid and is given by $\mu: \mathcal{T} \to \mathcal{W}, \mu(\mathcal{A}) = ...$
% \end{lemma}

% The final output of parsing is a normalized WIG. We define Welkin Canonical Form in the following fashion.
% \begin{definition}
% A WIG is in \textbf{Welkin Canonical Form (WCF)} if ...
% \end{definition}
% Based on this form, we have chosen a unique way to represent Welkin files. In particular, there is a representation under WIG (generalized) homotopies. We prove that there is a polynomial (or exponential?) algorithm to convert any WIG into WNF.

\section{General Application Behavior}

Note that all apparent structures may be adjusted under cryptomorphism.


\subsubsection*{Directives}
Each directive relies on the following components.
\begin{itemize}
  \item Input: processes an input or set of inputs.
  \item Parser: takes in a Welkin file and generates an AST,
  \item Validator: ensures that the AST is valid, raising an error that directly points to a violation,
  \item From here, an AST may be processed by three different means:
		\begin{itemize}
		  \item Recorder: takes the AST, converts it into a WIG in WCF, serializes the data,
		  \item Printer: displays some information provided in a Welkin file to the user.
		  \item Evaluator: evaluates, executes, or binds commands into Welkin units.
		\end{itemize}
\end{itemize}


\begin{itemize}
  \item self
  \item alias
  \item extend
  \item import
  \item parse
  \item validate
  \item output
  \item
\end{itemize}

% All official implementations may implement
% \begin{itemize}
% \item Base Welkin alone,
% \item Attribute and Base Welkin, or
% \item All three variants.
% \end{itemize}

% In each case, these variants must be implemented according to the welkin/bootstrap file. This file bootstraps the essential information from this standard. (See Section ?.? on a specification of this bootstrap)


% \begin{center}
%   \begin{tabular}{| c | c | c |}
% 	Directive & Definition & Example \\
% 	\hline
%     \texttt{import} & \makecell{Concatenates input \\ with current file} & Example \\
%     \texttt{import} & \makecell{Concatenates input \\ with current file} & Example \\
%     \texttt{self} & \makecell{Gets the \\ current graph} & Example \\
%     \texttt{input} & \makecell{Manage inputs} & Example \\
%     \texttt{parse} & \makecell{Manages parsing data} & Example \\
%     \texttt{validate} & \makecell{Concatenates input \\ with current file} & Example \\
%     \texttt{record} & \makecell{Concatenates input \\ with current file} & Example \\
%     \texttt{output} & \makecell{Manage outputs} & Example \\
%     \texttt{attribute} & \makecell{Concatenates input \\ with current file} & Example \\

%   \end{tabular}
% \end{center}

\subsubsection{Bootstrap}
\label{sec:bootstrap}
The welkin/bootstrap file faciliates the user API to Attribute and Binder Welkin. It is currently located at \url{https://github.com/AstralBearStudios/welkin} and is essential for creating a stable Welkin interpreter. Every official interpreter must follow this document first, and then be able to succesfully parse for the final (bootstrapped) attribute: \texttt{@attribute.}


% TODO: put below text somewhere else!
% Every Welkin interpreter must implement Base Welkin. To implement Attribute Welkin, the parser must be equipped with a parser for this variant, follow the API, and create a WIG for \texttt{attribute} (with all imported units in bootstrap).

% Finally, to support Binder Welkin, the interpreter must be written with a programming language that is Turing complete and can access system resources. The final command, $\texttt{@bind},$ is defined in terms of these notions. % TODO: is it definable in terms of these? Need to check and spell this out!



\subsection{Customization}

All Welkin files are infinitely customizable via the welkin config file, which is written in Attribute Welkin. Any attribute can be used, and other Welkin files can be imported. A base config file is required to customize a Welkin grammar. From there, configs can be arbitrarily nested to create and connect any desired (context-free) grammar, validator, recroder, and displayer.
% TODO: figure out best way to apply these directives to an entire folder or set of files
% TODO: replace itemzie with listing
% TODO: decide how to work with a folder (or folders) of config files
% \begin{itemize}
% 	\item Encoding
% 				\begin{itemize}
% 					% TODO: list major ascii versions/varieties. Need an official reference for this!
% 					\item Options: ascii, utf-8, utf-16, other
% 					\item In the case of other: we need to specify how to define an encoding. (We need a light-weight API for implementations)
% 				\end{itemize}
% 	\item Grammar
% 				\begin{itemize}
% 					\item Strength: bounded (only finitely nested graphs with a given nesting limit, no recursion), no-self (arbitrary nesting limit, but no recursion), full (recursion allowed)
% 					\item Customized: use a builtin template or custom welkin file. These can be used to change any part of the grammar, including adding keywords, the symbols used, adding new symbols, etc. Essentially, this will be a way to built new grammars from the original specification; we will need a separate parser for this (i.e., a parser of BNF/Welkin accepted notation).
% 				\end{itemize}

% 	\item (Optional) Language
% 				\begin{itemize}
% 					\item Defaults to English. Can be written in the writer's desired language (as long as it has been configured in Encoding above)
% 				\end{itemize}
% \end{itemize}

% In Welkin, we informally write the BNF above as follows:
% % TODO: explain unit notation (as it maybe clearer than the recursion below). In other words, mark arbitrary variables with the keyword unit
% % TODO: decide whether to introduce another arrow symbol for custom grammars.
% % While we have imposed few to no restrictions on custom grammars (besides being LALR), it may be the case
% % that multiple people want to use => for their own purpose. Is there a convenient way we can do this?
% \begin{quote}{\ttfamily \raggedright \noindent
% 	term -> \{ graph connection ident string\}\\
% 	graph -> \{\{ident \{\}\}->`\{'--term--`\}'\}\\
% 	connection -> \{term--connector--term\}\\
% 	connector -> \{edge arrow\}\\
% 	edge -> `--'\\
% 	arrow -> `->'\\
% 	ident -> CHAR*\\
% 	string -> ``'' CHAR* ``'' | `\`' CHAR* `\''
% }\end{quote}

\section{Core Algorithms}

\subsection{Graph Encoding}

\section{Conformance}
A program may be conformant Welkin interpreter for some variant. A program conforms to Base Welkin if each of the conditions hold:
\begin{itemize}
  \item It provides the Base Welkin pipeline
  \item It can store and retrieve any WIG.
\end{itemize}
Note

A program conforms Attribute Welkin if
\begin{itemize}
  \item It conforms to Base Welkin,
  \item It adds all
  \item It implements all attributes , and
  \item It bootstraps the last directive, \texttt{@attribute}, with the bootstrap folder.
\end{itemize}


Finally, a program is conformant to Binder Welkin if
\begin{itemize}
  \item It is conformant to Attribute Welkin, and
  \item It implements the following directives, as detailed in \ref{section:directives}:
  \begin{itemize}
    \item \texttt{@eval}
    \item \texttt{@exec}
    \item \texttt{@bind}
  \end{itemize}



\end{itemize}
